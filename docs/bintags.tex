\documentclass{article}

\usepackage{lmodern}
\PassOptionsToPackage{hyphens}{url} % url is loaded by hyperref
\usepackage{xcolor}
\definecolor{pink}{HTML}{EB877C}
\usepackage[colorlinks = true,
            urlcolor = pink]{hyperref}
\hypersetup{
            pdfborder={0 0 0},
            breaklinks=true}
            
\setlength{\emergencystretch}{3em}  % prevent overfull lines

% set default figure placement to htbp
\makeatletter
\def\fps@figure{htbp}
\makeatother

\title{Bin Tags}
\author{Angela Traugott\\CS2 Honors Project}
\date{Fall 2021}

\begin{document}

\maketitle

\section{Overview}\label{overview}

For my CS2 Honors Project, I wanted to make a system that could remotely
track orders being assembled, using OOP principles.~

\subsection{Criteria}\label{criteria}

From the ``Outcomes/Product'' section of the contract:~

\begin{quote}
I will have a minimum of two small devices with e-ink displays that will
show an order number and some information about the order. The devices
will be able to be charged in some way when they aren't in use (like
through solar power or induction). The devices will communicate with a
central host over BTLE (Bluetooth Low Energy) or LoRa (Long Range), and
the host can update each device with new information. I will be able to
push buttons on each device to update its order status. The host can
also see a list of all the devices and their order information,
including status.
\end{quote}

Bin Tags:

\begin{itemize}
\item
  {[} {]} send messages over LoRa
\item
  {[} {]} receive messages over LoRa
\item
  {[} {]} display order numbers and statuses
\item
  {[} {]} update displays freely (every time an order updates)
\item
  {[} {]} button press updates an order's status
\item
  {[} {]} relies on OOP (classes and methods)
\end{itemize}

Gateway:

\begin{itemize}
\item
  {[} {]} sends messages over LoRa
\item
  {[} {]} receives messages over LoRa
\item
  {[} {]} sends messages over MQTT
\item
  {[} {]} receives messages over MQTT
\item
  {[} {]} relies on OOP (classes and methods)
\end{itemize}

Central Host:

\begin{itemize}
\item
  {[} {]} sends messages over MQTT
\item
  {[} {]} receives messages over MQTT
\item
  {[} {]} stores boards/orders in a database
\item
  {[} {]} relies on OOP (classes and methods)
\end{itemize}

\subsection{Hardware}\label{hardware}

As expected, the hardware I was using changed over the course of the
first and second iterations. Listed below is the hardware I am using for
the final product.

Bin Tags:

\begin{itemize}
\item
  2 \href{https://www.adafruit.com/product/3857}{Adafruit Feather M4
  Express} boards~
\item
  2 \href{https://www.adafruit.com/product/3231}{Adafruit LoRa Radio
  FeatherWing} boards~
\item
  2 \href{https://www.adafruit.com/product/4777}{Adafruit 2.9" Greyscale
  eInk Displays}~
\end{itemize}

Gateway:

\begin{itemize}
\item
  1
  \href{https://www.raspberrypi.com/products/raspberry-pi-3-model-b/}{Raspberry
  Pi 3B}
\item
  1 \href{https://www.adafruit.com/product/4074}{Adafruit LoRa Radio
  Bonnet}
\end{itemize}

Central host:

\begin{itemize}
\item
  My own computer
\end{itemize}

\section{First Iteration}\label{first-iteration}

\subsection{Criteria}\label{criteria-1}

Bin Tags:

\begin{itemize}
\item
  {[}x{]} send messages over LoRa
\item
  {[} {]} receive messages over LoRa
\item
  {[}x{]} display order numbers and statuses
\item
  {[} {]} update displays freely (every time an order updates)
\item
  {[}x{]} button press updates an order's status
\item
  {[} {]} relies on OOP (classes and methods)
\end{itemize}

Gateway:

\begin{itemize}
\item
  {[} {]} sends messages over LoRa
\item
  {[}x{]} receives messages over LoRa
\item
  {[}x{]} sends messages over MQTT
\item
  {[} {]} receives messages over MQTT
\item
  {[} {]} relies on OOP (classes and methods)
\end{itemize}

Central Host:

\begin{itemize}
\item
  {[} {]} sends messages over MQTT
\item
  {[}x{]} receives messages over MQTT
\item
  {[} {]} stores boards/orders in a database
\item
  {[} {]} has command line interface
\item
  {[} {]} relies on OOP (classes and methods)
\end{itemize}

\subsection{Process and Outcome}\label{process-and-outcome}

The bin tags in the first iteration were successful in sending messages
over LoRa and displaying these messages simultaneously. However, the
libraries I was using were limited in a couple of ways.

First, the displays couldn't be updated more often than 3 minutes or the
program halted. This was to help prevent the displays from getting
permanently damaged (e-ink displays are apparently pretty fickle in this
regard), but it did get in the way of debugging while I was programming.
It sucked to have to wait 3 minutes before knowing if my code worked or
not. I didn't want to switch to more powerful OLED displays, because
those used up a lot more battery. Since I am planning to attach these to
bins for prolonged periods of time, I needed displays that could
function even if they weren't powered. So I thought this meant I needed
to find another library that could talk to the e-ink display without the
3-minute hassle.

Second (and definitely more important), I could only send messages over
LoRa; I couldn't receive any messages. This wasn't a problem with the
LoRa radio itself, just a problem with the library (I think the library
was written for sensors, which only need to send data). In this
iteration, the gateway I was using operated on a higher-level networking
layer for LoRa called LoRaWAN, which had more server-side support.
Unfortunately I hadn't found any other CircuitPython libraries that
could talk over LoRaWAN, so I thought I was going to have to switch over
to a much more versatile
\href{https://github.com/mcci-catena/arduino-lorawan}{C++ library}.

\section{Second Iteration}\label{second-iteration}

\subsection{Criteria}\label{criteria-2}

Bin Tags:

\begin{itemize}
\item
  {[}x{]} send messages over LoRa
\item
  {[}x{]} receive messages over LoRa
\item
  {[}x{]} display order numbers and statuses
\item
  {[} {]} update displays freely (every time an order updates)
\item
  {[}x{]} button press updates an order's status
\item
  {[}x{]} relies on OOP (classes and methods)
\end{itemize}

Gateway:

\begin{itemize}
\item
  {[}x{]} sends messages over LoRa
\item
  {[}x{]} receives messages over LoRa
\item
  {[} {]} sends messages over MQTT
\item
  {[} {]} receives messages over MQTT
\item
  {[} {]} relies on OOP (classes and methods)
\end{itemize}

Central Host:

\begin{itemize}
\item
  {[} {]} sends messages over MQTT
\item
  {[} {]} receives messages over MQTT
\item
  {[} {]} stores boards/orders in a database
\item
  {[} {]} has command line interface
\item
  {[} {]} relies on OOP (classes and methods)
\end{itemize}

\subsection{Process and Outcome}\label{process-and-outcome-1}

For my second iteration, I planned to try out the MCCI C++ library I
mentioned above. My goal was to be able to send and receive messages
over LoRaWAN, and to continue updating the display simultaneously. Since
the boards wouldn't be running CircuitPython anymore, I planned to use a
different library to talk to the displays, such as the
\href{https://github.com/adafruit/Adafruit_EPD}{Adafruit EPD} library.

After some grueling attempts to use the Arduino IDE to talk to the M4
boards, and trying to figure out how to use the MCCI library, I realized
that I wasn't making as much progress as I would have liked. I also
realized that I had a much more straightforward way of getting two
boards to talk to each other over LoRa, using
\href{https://github.com/adafruit/Adafruit_CircuitPython_RFM9x}{Adafruit's
RFM9x library}. Instead of relying on LoRaWAN, I was just going to deal
with the lower-level LoRa layer directly, and write more of my own
server-side code for the gateway and central host.

By the end of the second iteration I had a simple gateway working that
could receive/send messages to/from the bin tags over LoRa. I also
started incorporating more object-oriented programming ideas into the
code for the bin tags, starting with a \texttt{Message} and
\texttt{Order} class. I still had a 3-minute display refresh problem,
but I figured it wasn't worth switching over to C++ to get around that
problem.

\section{Third Iteration}\label{third-iteration}

\subsection{Criteria}\label{criteria-3}

Bin Tags:

\begin{itemize}
\item
  {[}x{]} send messages over LoRa
\item
  {[}x{]} receive messages over LoRa
\item
  {[}x{]} display order numbers and statuses
\item
  {[}x{]} update displays freely (every time an order updates)
\item
  {[}x{]} button press updates an order's status
\item
  {[}x{]} relies on OOP (classes and methods)
\end{itemize}

Gateway:

\begin{itemize}
\item
  {[}x{]} sends messages over LoRa
\item
  {[}x{]} receives messages over LoRa
\item
  {[}x{]} sends messages over MQTT
\item
  {[} {]} receives messages over MQTT
\item
  {[} {]} relies on OOP (classes and methods)
\end{itemize}

Central Host:

\begin{itemize}
\item
  {[} {]} sends messages over MQTT
\item
  {[}x{]} receives messages over MQTT
\item
  {[} {]} stores boards/orders in a database
\item
  {[} {]} has command line interface
\item
  {[} {]} relies on OOP (classes and methods)
\end{itemize}

\subsection{Process and Outcome}\label{process-and-outcome-2}

In my third iteration, I finally figured out how to get around the
180-second display refresh constraint that was built into the
\texttt{displayio} library on the bin tags.~

I added more code to the gateway so that it could send order updates
over MQTT. I also wrote a simple Go program that could subscribe to an
MQTT topic and see the messages that the gateway published.

\section{Final Iteration}\label{final-iteration}

\subsection{Criteria}\label{criteria-4}

Bin Tags:

\begin{itemize}
\item
  {[}x{]} send messages over LoRa
\item
  {[}x{]} receive messages over LoRa
\item
  {[}x{]} display order numbers and statuses
\item
  {[}x{]} update displays freely (every time an order updates)
\item
  {[}x{]} button press updates an order's status
\item
  {[}x{]} relies on OOP (classes and methods)
\end{itemize}

Gateway:

\begin{itemize}
\item
  {[}x{]} sends messages over LoRa
\item
  {[}x{]} receives messages over LoRa
\item
  {[}x{]} sends messages over MQTT
\item
  {[}x{]} receives messages over MQTT
\item
  {[} {]} relies on OOP (classes and methods)
\end{itemize}

Central Host:

\begin{itemize}
\item
  {[}x{]} sends messages over MQTT
\item
  {[}x{]} receives messages over MQTT
\item
  {[}x{]} stores boards/orders in a database
\item
  {[}x{]} has command line interface
\item
  {[}x{]} relies on OOP (classes and methods)
\end{itemize}

\subsection{Process and Outcome}\label{process-and-outcome-3}

I refactored the Python code on the bin tags to incorporate a lot more
OOP principles. Every physical component of the bin tag is now
represented by a class (\texttt{Button}, \texttt{Display},
\texttt{Radio}), and I also refactored my \texttt{Message} and
\texttt{Order} classes to make the code more streamlined. While the code
on the gateway still lacks classes and methods, it can be easily
refactored with a similar framework (\texttt{Message}, \texttt{Order},
and \texttt{Radio} classes).

Most importantly, I wrote the bulk of the central host code in this
iteration, using Go instead of Python. The central host is accessed by
two main commands:

\begin{itemize}
\item
  \texttt{serve} starts the server, which listens for MQTT messages and
  sends them, and it also has access to the database.
\item
  \texttt{reset} tells the client to update a board with a new order
  number, and the client in turn tells the server (over MQTT) to update
  that board. Since the client doesn't have direct access to the
  database and can talk to the server over MQTT, I can run the
  \texttt{serve} command on my own computer and the \texttt{reset}
  command on a different computer, and it would still work.
\end{itemize}

\section{Final Thoughts}\label{final-thoughts}

Overall, I am very happy with how this project turned out. Although it
doesn't use C++, I learned a lot in CS 2 that I was able to apply to my
project. My code has classes with methods and constructors, exception
handling with try-catch blocks, and pointers. Since Python is a
dynamically typed language, I don't need to use templates for it because
it is template-like by default. Go doesn't have classes or constructors,
but instead I can attach methods to structs, and I can write functions
that work like constructors (returning a struct from some given data).

\subsection{Going Forward}\label{going-forward}

Even though I fulfilled my honors contract, I have some goals for
continuing this project:

\begin{itemize}
\item
  Refactor gateway code to use OOP (not too difficult, since I already
  have a similar framework in the bin tag code)
\item
  Write a \texttt{list} command for the command line interface on the
  central host, to be able to list all of the orders without direct
  access to the database file (also not too difficult, since I already
  have code that accesses the database)
\item
  Design and 3D print holders for the bin tags, so that they can be
  attached to bins
\item
  Design and 3D print a case for the gateway
\end{itemize}

\end{document}

